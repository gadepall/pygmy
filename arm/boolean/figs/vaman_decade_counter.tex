%\documentclass{article}

%\usepackage[latin1]{inputenc}
%\usepackage{tikz}
%\usetikzlibrary{shapes,arrows}

%%%%<
%\usepackage{verbatim}
%\usepackage[active,tightpage]{preview}
%\PreviewEnvironment{tikzpicture}
%\setlength\PreviewBorder{5pt}%
%%%%>

%\begin{comment}
%:Title: Simple flow chart
%:Tags: Diagrams

%With PGF/TikZ you can draw flow charts with relative ease. This flow chart from [1]_
%outlines an algorithm for identifying the parameters of an autonomous underwater vehicle model. 

%Note that relative node
%placement has been used to avoid placing nodes explicitly. This feature was
%introduced in PGF/TikZ >= 1.09.

%.. [1] Bossley, K.; Brown, M. & Harris, C. Neurofuzzy identification of an autonomous underwater vehicle `International Journal of Systems Science`, 1999, 30, 901-913 


%\end{comment}


%\begin{document}
%\pagestyle{empty}


% Define block styles
\tikzstyle{decision} = [diamond, draw, fill=blue!20, 
    text width=4.5em, text badly centered, node distance=3cm, inner sep=0pt]
%\tikzstyle{block} = [rectangle, draw, fill=blue!20, 
%    text width=5em, text centered, rounded corners, minimum height=4em]
\tikzstyle{block} = [rectangle, draw, 
    text width=5em, text centered, rounded corners, minimum height=4em]

\tikzstyle{line} = [draw, -latex']
\tikzstyle{cloud} = [draw, ellipse,fill=red!20, node distance=3cm,
    minimum height=2em]
    
\begin{tikzpicture}[node distance = 3cm, auto]
    % Place nodes
%    \node [block] (init) {Incrementing Decoder};
    \node [block] (init) {परवर्ती गूढ़वाचक};
%    \node [cloud, left of=init] (expert) {expert};
%    \node [cloud, right of=init] (system) {system};
%    \node [block, below of=init, node distance = 4cm] (identify) {Display Decoder};
    \node [block, below of=init, node distance = 4cm] (identify) {प्रदर्शी गूढ़वाचक};
%    \node [block, below of=identify ] (evaluate) {Seven-Segment Display};
    \node [block, below of=identify ] (evaluate) {सप्तांश प्रदर्शी};
%    \node [block, right of=identify, node distance = 4cm] (delay) {Delay};
     %\node [block, (4,-3)] (q1) {Delay};
%	\node at (4,-2)[block] (delay) {Delay};
	\node at (4,-2)[block] (delay) {अतिकाल};
\begin{scope}[->,>=latex]
    \foreach \i in {-3,-1,1,3}
    { 
%      \draw[->] ([yshift=\i * 0.2 cm]identify.east) -- ([yshift=\i * 0.2 cm]delay.west) ;
      \draw[->] ([xshift=\i * 0.2 cm]delay.north) |- ([yshift=\i * 0.2 cm]init.east) ;
      \draw[->] ([xshift=\i * 0.2 cm]init.south) -- ([xshift=\i * 0.2 cm]identify.north) ;
       \draw node at (\i * 0.2,-2+\i * 0.2) { \textbullet} ;
       \draw[->] (\i * 0.2,-2+\i * 0.2) -- ([yshift=\i * 0.2 cm]delay.west) ;
      
    }
\foreach \i in {-3,...,3}
    { 
      \draw[->] ([xshift=\i * 0.35 cm]identify.south) -- ([xshift=\i * 0.35 cm]evaluate.north) ;
    }
\foreach [count=\i] \j in {a,b,...,g}{
            \node (\i) at ( 1.6-\i * 0.35, -5.5) {\j} ;
            }
\foreach [count=\i] \j in {A,B,C,D}{
            \node (\i) at ( 0.8-\i * 0.4, -1.0-\i*0.4) {\j} ;
            }

\foreach [count=\i] \j in {W,X,Y,Z}{
            \node (\i) at ( 1.6, 1.2-\i*0.4) {\j} ;
            }
    
\end{scope}

 %   \node [block, left of=evaluate, node distance=3cm] (update) {update model};
  %  \node [decision, below of=evaluate] (decide) {is best candidate better?};
%    \node [block, below of=decide, node distance=3cm] (stop) {stop};
    % Draw edges
%    \path [line] (init) -- (identify);
    \path [line] (identify) -- (evaluate);
%    \path [line] (evaluate) -- (decide);
  %  \path [line] (decide) -| node [near start] {yes} (update);
   % \path [line] (update) |- (identify);
 %   \path [line] (decide) -- node {no}(stop);
%    \path [line,dashed] (expert) -- (init);
%    \path [line,dashed] (system) -- (init);
%    \path [line,dashed] (system) |- (evaluate);
\end{tikzpicture}
%}

%\end{document}
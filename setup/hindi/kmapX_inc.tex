\renewcommand{\theequation}{\theenumi}
\renewcommand{\thefigure}{\theenumi}
\begin{enumerate}[label=\thesubsection.\arabic*.,ref=\thesubsection.\theenumi]
\numberwithin{equation}{enumi}
\numberwithin{figure}{enumi}
\numberwithin{table}{enumi}

\item आकृति \ref{fig:inc_kmapX_A} के द्वारा $A$ के व्यंजक को व्युत्पन्न करें।

\solution

\begin{align}
\label{eq:kmapX_disp_A}
A &= W^{\prime}
\end{align}
%

\begin{figure}[!ht]
\centering
\resizebox{\columnwidth}{!} {
\input{figs/inc/kmapX/A.tex}
}
\caption{$A$ का निर्गुण विवक्षक कृत क-मानचित्र।}
\label{fig:inc_kmapX_A}
\end{figure}
%

\item आकृति \ref{fig:inc_kmapX_B} के द्वारा $B$ के व्यंजक को व्युत्पन्न करें।

\solution

\begin{align}
\label{eq:kmapX_disp_B}
B &= WX^{\prime}Z^{\prime}+W^{\prime}X
\end{align}
%

\begin{figure}[!ht]
\centering
\resizebox{\columnwidth}{!} {
\input{figs/inc/kmapX/B.tex}
}
\caption{$B$ का निर्गुण विवक्षक कृत क-मानचित्र।}
\label{fig:inc_kmapX_B}
\end{figure}
%
\item आकृति \ref{fig:inc_kmapX_C} के द्वारा $C$ के व्यंजक को व्युत्पन्न करें।

\solution

\begin{align}
\label{eq:kmapX_disp_C}
C &= X^{\prime}Y+W^{\prime}Y++WXY^{\prime}
\end{align}
%

\begin{figure}[!ht]
\centering
\resizebox{\columnwidth}{!} {
\input{figs/inc/kmapX/C.tex}
}
\caption{$C$ का निर्गुण विवक्षक कृत क-मानचित्र।}
\label{fig:inc_kmapX_C}
\end{figure}
%
%
\item आकृति \ref{fig:inc_kmapX_D} के द्वारा $D$ के व्यंजक को व्युत्पन्न करें।

\solution

\begin{align}
\label{eq:kmapX_disp_D}
D &= WXY+W^{\prime}Z
\end{align}
%

\begin{figure}[!ht]
\centering
\resizebox{\columnwidth}{!} {
\input{figs/inc/kmapX/D.tex}
}
\caption{$D$ का निर्गुण विवक्षक कृत क-मानचित्र।}
\label{fig:inc_kmapX_D}
\end{figure}
%

\end{enumerate}

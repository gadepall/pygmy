निम्न  चरणों में सारणी \ref{table:disp_dec}  एवं के-मानचित्र के द्वारा $a,b,c,d,e,f,g$ के न्यूनतम व्यंजक को व्युत्पन्न किया जाएगा.

\renewcommand{\theequation}{\theenumi}
\renewcommand{\thefigure}{\theenumi}
\begin{enumerate}[label=\thesubsection.\arabic*.,ref=\thesubsection.\theenumi]
\numberwithin{equation}{enumi}
\numberwithin{figure}{enumi}
\numberwithin{table}{enumi}

\item आकृति \ref{fig:disp_kmap_a} के द्वारा $a$ के व्यंजक को व्युत्पन्न करें.


\solution

\begin{align}
\label{eq:kmap_disp_a}
a &= A B^{\prime}C^{\prime} D^{\prime} + A^{\prime}B^{\prime} C D^{\prime}
\\
&= B^{\prime}D^{\prime}\brak{A C^{\prime}+ A^{\prime}C}
\\
&=B^{\prime}D^{\prime}\brak{A \oplus  C}
\end{align}
%
\begin{figure}[!ht]
\centering
\resizebox{\columnwidth}{!} {
\input{figs/disp/kmap/a.tex}
}
\caption{$a$ का क-मानचित्र.}
\label{fig:disp_kmap_a}
\end{figure}
%
$\oplus$ संक्रिया की परिभाषा सारणी \ref{table:xor}  में उपलब्ध     है.

\begin{table}[!ht]
\centering
\begin{tabular}{|l|l|l|}
\hline
$A$ & $C$ & $A\oplus C$ \\ \hline
0   & 0   & 0          \\ \hline
0   & 1   & 1          \\ \hline
1   & 0   & 1          \\ \hline
1   & 1   & 0          \\ \hline
\end{tabular}
\caption{$\oplus$ की परिभाषा.}
\label{table:xor}
\end{table}

\item आकृति \ref{fig:disp_kmap_b} के द्वारा $b$ के व्यंजक को व्युत्पन्न करें.


\solution

\begin{align}
\label{eq:kmap_disp_b}
b &= A B^{\prime}C D^{\prime} + A^{\prime}B C D^{\prime}
\\
&=C D^{\prime}\brak{A \oplus B}
\end{align}
%


\begin{figure}[!ht]
\centering
\resizebox{\columnwidth}{!} {
\input{figs/disp/kmap/b.tex}
}
\caption{$b$ का क-मानचित्र।}
\label{fig:disp_kmap_b}
\end{figure}
%
\item आकृति \ref{fig:disp_kmap_c} के द्वारा $c$ के व्यंजक को व्युत्पन्न करें।


\solution

\begin{align}
\label{eq:kmap_disp_c}
c &=  A^{\prime}B C^{\prime} D^{\prime}
\end{align}
%
\begin{figure}[!ht]
\centering
\resizebox{\columnwidth}{!} {
\input{figs/disp/kmap/c.tex}
}
\caption{$c$ का क-मानचित्र।}
\label{fig:disp_kmap_c}
\end{figure}
%
\item  आकृति \ref{fig:disp_kmap_d} के द्वारा $d$ के व्यंजक को व्युत्पन्न करें।
\\
\solution


\begin{align}
\label{eq:kmap_disp_d}
d=AB^{\prime}C^{\prime}+A^{\prime}B^{\prime}CD^{\prime}+ABCD^{\prime}
\end{align}
%
\begin{figure}[!ht]
\centering
\resizebox{\columnwidth}{!} {
\input{figs/disp/kmap/d.tex}
}
\caption{$d$ का क-मानचित्र।}
\label{fig:disp_kmap_d}
\end{figure}
\item आकृति \ref{fig:disp_kmap_e} के द्वारा $e$ के व्यंजक को व्युत्पन्न करें।
%

\begin{align}
\label{eq:kmap_disp_e}
e=AD^{\prime}+B^{\prime}CD^{\prime}+AB^{\prime}C^{\prime}
\end{align}
%
\begin{figure}[!ht]
\centering
\resizebox{\columnwidth}{!} {
\input{figs/disp/kmap/e.tex}
}
\caption{$e$ का क-मानचित्र।}
\label{fig:disp_kmap_e}
\end{figure}
%
\item आकृति \ref{fig:disp_kmap_f} के द्वारा $f$ के व्यंजक को व्युत्पन्न करें।
%
\begin{align}
\label{eq:kmap_disp_f}
f= BC^{\prime}D^{\prime} + AB^{\prime}C^{\prime}D^{\prime} + AB^{\prime}CD
\end{align}
%
\begin{figure}[!ht]
\centering
\resizebox{\columnwidth}{!} {
\input{figs/disp/kmap/f.tex}
}
\caption{$f$ का क-मानचित्र।}
\label{fig:disp_kmap_f}
\end{figure}

%
\item आकृति \ref{fig:disp_kmap_g} के द्वारा $g$ के व्यंजक को व्युत्पन्न करें।
%
\begin{align}
\label{eq:kmap_disp_g}
g = B^{\prime}C^{\prime}D^{\prime}+ABCD^{\prime}
\end{align}
%
\begin{figure}[!ht]
\centering
\resizebox{\columnwidth}{!} {
\input{figs/disp/kmap/g.tex}
}
\caption{$g$ का क-मानचित्र।}
\label{fig:disp_kmap_g}
\end{figure}
\end{enumerate}
%
%

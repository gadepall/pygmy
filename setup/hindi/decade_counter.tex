\subsection{प्रस्तावना}
दशक गणित्र  एक  तुलयात्मक  परिपथ है जो अनुक्रम 0-9 की समान अतिकाल से निरन्तर   गणना करता है।  इसका खंड आरेख 
आकृति. \ref{fig:dec_counter} में उपलब्ध है।  

\begin{figure}[!h]
\resizebox {\columnwidth} {!} {
\input{./figs/decade/decade_counter_hindi}
%\subsection{प्रस्तावना}
दशक गणित्र  एक  तुलयात्मक  परिपथ है जो अनुक्रम 0-9 की समान अतिकाल से निरन्तर   गणना करता है।  इसका खंड आरेख 
आकृति. \ref{fig:dec_counter} में उपलब्ध है।  

\begin{figure}[!h]
\resizebox {\columnwidth} {!} {
\input{./figs/decade/decade_counter_hindi}
%\subsection{प्रस्तावना}
दशक गणित्र  एक  तुलयात्मक  परिपथ है जो अनुक्रम 0-9 की समान अतिकाल से निरन्तर   गणना करता है।  इसका खंड आरेख 
आकृति. \ref{fig:dec_counter} में उपलब्ध है।  

\begin{figure}[!h]
\resizebox {\columnwidth} {!} {
\input{./figs/decade/decade_counter_hindi}
%\subsection{प्रस्तावना}
दशक गणित्र  एक  तुलयात्मक  परिपथ है जो अनुक्रम 0-9 की समान अतिकाल से निरन्तर   गणना करता है।  इसका खंड आरेख 
आकृति. \ref{fig:dec_counter} में उपलब्ध है।  

\begin{figure}[!h]
\resizebox {\columnwidth} {!} {
\input{./figs/decade/decade_counter_hindi}
%\input{./figs/decade/decade_counter}
}
\caption{दशक गणित्र का खंड आरेख}
\label{fig:dec_counter}
\end{figure}
%
\subsection{परिमित अवस्था यंत्र}
%
\ref{fig:dec_counter} का परिमित अवस्था यंत्र आरेख आकृति. \ref{fig:fsm_counter} में उपलब्ध है।  $s_0$ वह अवस्था है जहां परवर्ती गूढ़वाचक का आगत मूल्य  0 है।  दशक गणित्र  की अवस्थान्तरण सारणी   \ref{table:counter_decoder}  में प्रस्तुत हैं।  इसमें वर्तमान अवस्था चर  का प्रबोधन  $W,X,Y,Z$ से है एवं आगामी अवस्था चर  $A,B,C,D$ द्वारा प्रबोधित है।
\begin{figure}[!h]
\centering
\resizebox {\columnwidth} {!} {
\input{./figs/decade/fsm_counter_hindi}
}
\caption{दशक गणित्र की अवस्थायें।}
\label{fig:fsm_counter}
\end{figure}
%
\subsection{परिमित अवस्था यंत्र के द्वारा दशक गणित्र का अभिकल्प}

 आकृति. \ref{fig:dff} में D-द्विविध के व्यूह से दशक गणित्र का अभिकल्प उपलब्ध है।  यहाँ D-द्विविध आकृति \ref{fig:dec_counter} में अतिकाल खंड को  आकृति \ref{fig:dff} के परिपथ में कार्यान्वित करता है। D-द्विविध आगत मूल्य को घड़ी के आवर्त समय के पश्चात निर्गमन
करता है।
\begin{figure}[!h]
\resizebox {\columnwidth} {!} {
\input{./figs/decade/dff_hindi}
}
\caption{D-द्विविध द्वारा परिमित अवस्था यंत्र का कार्यान्वयन।}
\label{fig:dff}
\end{figure}
%
आकृति \ref{fig:dff} में यंत्रोपवस्तु मूल्य  सारणी \ref{table:fsm_counter} में प्रदत्त है।
\begin{table}[!h]
\resizebox {\columnwidth} {!} {
\input{./tables/flip_flop_hindi}
}
\caption{यंत्रोपवस्तु मूल्य।}
\label{table:fsm_counter}
\end{table}

उपरोक्त विधान से पूर्ववर्ती गूढ़वाचक का अभिकल्प करें।

%\begin{equation}
%\text{No. of D Flip-Flops} = \ceil{\log_{2}\brak{\text{No. of States}}}
%\end{equation}
%For the FSM in Fig. \ref{fig:fsm_counter}, the number of states is 9, hence the number flipflops required = 4.  
%\begin{problem}
%Design a decade down counter (counts from 9 to 0 repeatedly) using an FSM.  
%\end{problem}

\end{document}



}
\caption{दशक गणित्र का खंड आरेख}
\label{fig:dec_counter}
\end{figure}
%
\subsection{परिमित अवस्था यंत्र}
%
\ref{fig:dec_counter} का परिमित अवस्था यंत्र आरेख आकृति. \ref{fig:fsm_counter} में उपलब्ध है।  $s_0$ वह अवस्था है जहां परवर्ती गूढ़वाचक का आगत मूल्य  0 है।  दशक गणित्र  की अवस्थान्तरण सारणी   \ref{table:counter_decoder}  में प्रस्तुत हैं।  इसमें वर्तमान अवस्था चर  का प्रबोधन  $W,X,Y,Z$ से है एवं आगामी अवस्था चर  $A,B,C,D$ द्वारा प्रबोधित है।
\begin{figure}[!h]
\centering
\resizebox {\columnwidth} {!} {
\input{./figs/decade/fsm_counter_hindi}
}
\caption{दशक गणित्र की अवस्थायें।}
\label{fig:fsm_counter}
\end{figure}
%
\subsection{परिमित अवस्था यंत्र के द्वारा दशक गणित्र का अभिकल्प}

 आकृति. \ref{fig:dff} में D-द्विविध के व्यूह से दशक गणित्र का अभिकल्प उपलब्ध है।  यहाँ D-द्विविध आकृति \ref{fig:dec_counter} में अतिकाल खंड को  आकृति \ref{fig:dff} के परिपथ में कार्यान्वित करता है। D-द्विविध आगत मूल्य को घड़ी के आवर्त समय के पश्चात निर्गमन
करता है।
\begin{figure}[!h]
\resizebox {\columnwidth} {!} {
\input{./figs/decade/dff_hindi}
}
\caption{D-द्विविध द्वारा परिमित अवस्था यंत्र का कार्यान्वयन।}
\label{fig:dff}
\end{figure}
%
आकृति \ref{fig:dff} में यंत्रोपवस्तु मूल्य  सारणी \ref{table:fsm_counter} में प्रदत्त है।
\begin{table}[!h]
\resizebox {\columnwidth} {!} {
\input{./tables/flip_flop_hindi}
}
\caption{यंत्रोपवस्तु मूल्य।}
\label{table:fsm_counter}
\end{table}

उपरोक्त विधान से पूर्ववर्ती गूढ़वाचक का अभिकल्प करें।

%\begin{equation}
%\text{No. of D Flip-Flops} = \ceil{\log_{2}\brak{\text{No. of States}}}
%\end{equation}
%For the FSM in Fig. \ref{fig:fsm_counter}, the number of states is 9, hence the number flipflops required = 4.  
%\begin{problem}
%Design a decade down counter (counts from 9 to 0 repeatedly) using an FSM.  
%\end{problem}

\end{document}



}
\caption{दशक गणित्र का खंड आरेख}
\label{fig:dec_counter}
\end{figure}
%
\subsection{परिमित अवस्था यंत्र}
%
\ref{fig:dec_counter} का परिमित अवस्था यंत्र आरेख आकृति. \ref{fig:fsm_counter} में उपलब्ध है।  $s_0$ वह अवस्था है जहां परवर्ती गूढ़वाचक का आगत मूल्य  0 है।  दशक गणित्र  की अवस्थान्तरण सारणी   \ref{table:counter_decoder}  में प्रस्तुत हैं।  इसमें वर्तमान अवस्था चर  का प्रबोधन  $W,X,Y,Z$ से है एवं आगामी अवस्था चर  $A,B,C,D$ द्वारा प्रबोधित है।
\begin{figure}[!h]
\centering
\resizebox {\columnwidth} {!} {
\input{./figs/decade/fsm_counter_hindi}
}
\caption{दशक गणित्र की अवस्थायें।}
\label{fig:fsm_counter}
\end{figure}
%
\subsection{परिमित अवस्था यंत्र के द्वारा दशक गणित्र का अभिकल्प}

 आकृति. \ref{fig:dff} में D-द्विविध के व्यूह से दशक गणित्र का अभिकल्प उपलब्ध है।  यहाँ D-द्विविध आकृति \ref{fig:dec_counter} में अतिकाल खंड को  आकृति \ref{fig:dff} के परिपथ में कार्यान्वित करता है। D-द्विविध आगत मूल्य को घड़ी के आवर्त समय के पश्चात निर्गमन
करता है।
\begin{figure}[!h]
\resizebox {\columnwidth} {!} {
\input{./figs/decade/dff_hindi}
}
\caption{D-द्विविध द्वारा परिमित अवस्था यंत्र का कार्यान्वयन।}
\label{fig:dff}
\end{figure}
%
आकृति \ref{fig:dff} में यंत्रोपवस्तु मूल्य  सारणी \ref{table:fsm_counter} में प्रदत्त है।
\begin{table}[!h]
\resizebox {\columnwidth} {!} {
\input{./tables/flip_flop_hindi}
}
\caption{यंत्रोपवस्तु मूल्य।}
\label{table:fsm_counter}
\end{table}

उपरोक्त विधान से पूर्ववर्ती गूढ़वाचक का अभिकल्प करें।

%\begin{equation}
%\text{No. of D Flip-Flops} = \ceil{\log_{2}\brak{\text{No. of States}}}
%\end{equation}
%For the FSM in Fig. \ref{fig:fsm_counter}, the number of states is 9, hence the number flipflops required = 4.  
%\begin{problem}
%Design a decade down counter (counts from 9 to 0 repeatedly) using an FSM.  
%\end{problem}

\end{document}



}
\caption{दशक गणित्र का खंड आरेख}
\label{fig:dec_counter}
\end{figure}
%
\subsection{परिमित अवस्था यंत्र}
%
\ref{fig:dec_counter} का परिमित अवस्था यंत्र आरेख आकृति. \ref{fig:fsm_counter} में उपलब्ध है।  $s_0$ वह अवस्था है जहां परवर्ती गूढ़वाचक का आगत मूल्य  0 है।  दशक गणित्र  की अवस्थान्तरण सारणी   \ref{table:counter_decoder}  में प्रस्तुत हैं।  इसमें वर्तमान अवस्था चर  का प्रबोधन  $W,X,Y,Z$ से है एवं आगामी अवस्था चर  $A,B,C,D$ द्वारा प्रबोधित है।
\begin{figure}[!h]
\centering
\resizebox {\columnwidth} {!} {
\input{./figs/decade/fsm_counter_hindi}
}
\caption{दशक गणित्र की अवस्थायें।}
\label{fig:fsm_counter}
\end{figure}
%
\subsection{परिमित अवस्था यंत्र के द्वारा दशक गणित्र का अभिकल्प}

 आकृति. \ref{fig:dff} में D-द्विविध के व्यूह से दशक गणित्र का अभिकल्प उपलब्ध है।  यहाँ D-द्विविध आकृति \ref{fig:dec_counter} में अतिकाल खंड को  आकृति \ref{fig:dff} के परिपथ में कार्यान्वित करता है। D-द्विविध आगत मूल्य को घड़ी के आवर्त समय के पश्चात निर्गमन
करता है।
\begin{figure}[!h]
\resizebox {\columnwidth} {!} {
\input{./figs/decade/dff_hindi}
}
\caption{D-द्विविध द्वारा परिमित अवस्था यंत्र का कार्यान्वयन।}
\label{fig:dff}
\end{figure}
%
आकृति \ref{fig:dff} में यंत्रोपवस्तु मूल्य  सारणी \ref{table:fsm_counter} में प्रदत्त है।
\begin{table}[!h]
\resizebox {\columnwidth} {!} {
\input{./tables/flip_flop_hindi}
}
\caption{यंत्रोपवस्तु मूल्य।}
\label{table:fsm_counter}
\end{table}

उपरोक्त विधान से पूर्ववर्ती गूढ़वाचक का अभिकल्प करें।

%\begin{equation}
%\text{No. of D Flip-Flops} = \ceil{\log_{2}\brak{\text{No. of States}}}
%\end{equation}
%For the FSM in Fig. \ref{fig:fsm_counter}, the number of states is 9, hence the number flipflops required = 4.  
%\begin{problem}
%Design a decade down counter (counts from 9 to 0 repeatedly) using an FSM.  
%\end{problem}

\end{document}



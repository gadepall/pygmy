निम्न  चरणों में सारणी \ref{table:disp_dec}  एवं  निर्गुण  विवक्षकृत  क-मानचित्र के द्वारा $a,b,c,d,e,f,g$ के न्यूनतम व्यंजक को व्युत्पन्न किया जाएगा.

\renewcommand{\theequation}{\theenumi}
\renewcommand{\thefigure}{\theenumi}
\begin{enumerate}[label=\thesubsection.\arabic*.,ref=\thesubsection.\theenumi]
\numberwithin{equation}{enumi}
\numberwithin{figure}{enumi}
\numberwithin{table}{enumi}

\item आकृति \ref{fig:disp_kmapX_a} के द्वारा $a$ के व्यंजक को व्युत्पन्न करें.


\solution

\begin{align}
\label{eq:kmapX_disp_a}
a &= A B^{\prime}C^{\prime} D^{\prime} + A^{\prime}B^{\prime} C 
\end{align}
%
\begin{figure}[!ht]
\centering
\resizebox{\columnwidth}{!} {
\input{figs/disp/kmapX/a.tex}
}
\caption{$a$ का निर्गुण  विवक्षकृत क-मानचित्र.}
\label{fig:disp_kmapX_a}
\end{figure}
%

\item आकृति \ref{fig:disp_kmapX_b} के द्वारा $b$ के व्यंजक को व्युत्पन्न करें.


\solution

\begin{align}
\label{eq:kmapX_disp_b}
b &= A B^{\prime}C  + A^{\prime}B C 
\\
&= C (A\oplus B)
\end{align}
%


\begin{figure}[!ht]
\centering
\resizebox{\columnwidth}{!} {
\input{figs/disp/kmapX/b.tex}
}
\caption{$b$ का निर्गुण  विवक्षकृत क-मानचित्र।}
\label{fig:disp_kmapX_b}
\end{figure}
%
\item आकृति \ref{fig:disp_kmapX_c} के द्वारा $c$ के व्यंजक को व्युत्पन्न करें।


\solution

\begin{align}
\label{eq:kmapX_disp_c}
c &=  A^{\prime}B C^{\prime}
\end{align}
%
\begin{figure}[!ht]
\centering
\resizebox{\columnwidth}{!} {
\input{figs/disp/kmapX/c.tex}
}
\caption{$c$ का निर्गुण  विवक्षकृत क-मानचित्र।}
\label{fig:disp_kmapX_c}
\end{figure}
%
\item  आकृति \ref{fig:disp_kmapX_d} के द्वारा $d$ के व्यंजक को व्युत्पन्न करें।
\\
\solution


\begin{align}
\label{eq:kmapX_disp_d}
d=AB^{\prime}C^{\prime}+A^{\prime}B^{\prime}C+ABC
\end{align}
%
\begin{figure}[!ht]
\centering
\resizebox{\columnwidth}{!} {
\input{figs/disp/kmapX/d.tex}
}
\caption{$d$ का निर्गुण  विवक्षकृत क-मानचित्र।}
\label{fig:disp_kmapX_d}
\end{figure}
\item आकृति \ref{fig:disp_kmapX_e} के द्वारा $e$ के व्यंजक को व्युत्पन्न करें।
%

\begin{align}
\label{eq:kmapX_disp_e}
e=A+B^{\prime}C
\end{align}
%
\begin{figure}[!ht]
\centering
\resizebox{\columnwidth}{!} {
\input{figs/disp/kmapX/e.tex}
}
\caption{$e$ का निर्गुण  विवक्षकृत क-मानचित्र।}
\label{fig:disp_kmapX_e}
\end{figure}
%
\item आकृति \ref{fig:disp_kmapX_f} के द्वारा $f$ के व्यंजक को व्युत्पन्न करें।
%
\begin{align}
\label{eq:kmapX_disp_f}
f= AB + AC^{\prime}D^{\prime} + BC^{\prime}
\end{align}
%
\begin{figure}[!ht]
\centering
\resizebox{\columnwidth}{!} {
\input{figs/disp/kmapX/f.tex}
}
\caption{$f$ का निर्गुण  विवक्षकृत क-मानचित्र।}
\label{fig:disp_kmapX_f}
\end{figure}

%
\item आकृति \ref{fig:disp_kmapX_g} के द्वारा $g$ के व्यंजक को व्युत्पन्न करें।
%
\begin{align}
\label{eq:kmapX_disp_g}
g = B^{\prime}C^{\prime}D^{\prime}+ABC
\end{align}
%
\begin{figure}[!ht]
\centering
\resizebox{\columnwidth}{!} {
\input{figs/disp/kmapX/g.tex}
}
\caption{$g$ का निर्गुण  विवक्षकृत क-मानचित्र।}
\label{fig:disp_kmapX_g}
\end{figure}
\end{enumerate}
%
%

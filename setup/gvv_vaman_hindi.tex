\documentclass[journal,12pt,twocolumn]{IEEEtran}
%
\usepackage{setspace}
\usepackage{gensymb}
%\doublespacing
\singlespacing

%\usepackage{graphicx}
%\usepackage{amssymb}
%\usepackage{relsize}
\usepackage[cmex10]{amsmath}
\usepackage{siunitx}
%\usepackage{amsthm}
%\interdisplaylinepenalty=2500
%\savesymbol{iint}
%\usepackage{txfonts}
%\restoresymbol{TXF}{iint}
%\usepackage{wasysym}
\usepackage{amsthm}
\usepackage{iithtlc}
\usepackage{mathrsfs}
\usepackage{txfonts}
\usepackage{stfloats}
\usepackage{steinmetz}
\usepackage{supertabular}
%\usepackage{bm}
\usepackage{cite}
\usepackage{cases}
\usepackage{subfig}
%\usepackage{xtab}
\usepackage{longtable}
\usepackage{multirow}
%\usepackage{algorithm}
%\usepackage{algpseudocode}
\usepackage{enumitem}
\usepackage{mathtools}
\usepackage{tikz}
\usepackage{circuitikz}
\usepackage{verbatim}
\usepackage{tfrupee}
\usepackage[breaklinks=true]{hyperref}
%\usepackage{stmaryrd}
\usepackage{tkz-euclide} % loads  TikZ and tkz-base
%\usetkzobj{all}
\usetikzlibrary{calc,math}
\usetikzlibrary{fadings}
\usepackage{listings}
    \usepackage{color}                                            %%
    \usepackage{array}                                            %%
    \usepackage{longtable}                                        %%
    \usepackage{calc}                                             %%
    \usepackage{multirow}                                         %%
    \usepackage{hhline}                                           %%
    \usepackage{ifthen}                                           %%
  %optionally (for landscape tables embedded in another document): %%
    \usepackage{lscape}     
\usepackage{multicol}
\usepackage{chngcntr}
\usepackage{blkarray}
\usepackage{karnaugh-map}
\usepackage{fontspec}
\usepackage[intoc]{nomencl}
\makenomenclature

%\usetikzlibrary{arrows, shapes.gates.logic.US, calc}
\usetikzlibrary{arrows,shapes.gates.logic.US,shapes.gates.logic.IEC,calc}
\setmainfont{Sanskrit_2003.ttf}
%\setmainfont{Nakula.ttf}
%\setmainfont{Lohit-Devanagari.ttf}


%\usepackage{enumerate}

%\usepackage{wasysym}
%\newcounter{MYtempeqncnt}
\DeclareMathOperator*{\Res}{Res}
%\renewcommand{\baselinestretch}{2}
\renewcommand\thesection{\arabic{section}}
\renewcommand\thesubsection{\thesection.\arabic{subsection}}
\renewcommand\thesubsubsection{\thesubsection.\arabic{subsubsection}}

\renewcommand\thesectiondis{\arabic{section}}
\renewcommand\thesubsectiondis{\thesectiondis.\arabic{subsection}}
\renewcommand\thesubsubsectiondis{\thesubsectiondis.\arabic{subsubsection}}

% correct bad hyphenation here
\hyphenation{op-tical net-works semi-conduc-tor}
\def\inputGnumericTable{}                                 %%


\lstset{
%language=shell,
%language = Prolog,
frame=single, 
breaklines=true,
%showstringspaces=false,
columns=fullflexible
literate = {-}{-}1
}
%\lstset{
%language=tex,
%frame=single, 
%breaklines=true
%}

\begin{document}
%


\newtheorem{theorem}{Theorem}[section]
\newtheorem{problem}{Problem}
\newtheorem{proposition}{Proposition}[section]
\newtheorem{lemma}{Lemma}[section]
\newtheorem{corollary}[theorem]{Corollary}
\newtheorem{example}{Example}[section]
\newtheorem{definition}[problem]{Definition}
%\newtheorem{thm}{Theorem}[section] 
%\newtheorem{defn}[thm]{Definition}
%\newtheorem{algorithm}{Algorithm}[section]
%\newtheorem{cor}{Corollary}
\newcommand{\BEQA}{\begin{eqnarray}}
\newcommand{\EEQA}{\end{eqnarray}}
\newcommand{\define}{\stackrel{\triangle}{=}}

\bibliographystyle{IEEEtran}
%\bibliographystyle{ieeetr}


\providecommand{\mbf}{\mathbf}
\providecommand{\pr}[1]{\ensuremath{\Pr\left(#1\right)}}
\providecommand{\qfunc}[1]{\ensuremath{Q\left(#1\right)}}
\providecommand{\sbrak}[1]{\ensuremath{{}\left[#1\right]}}
\providecommand{\lsbrak}[1]{\ensuremath{{}\left[#1\right.}}
\providecommand{\rsbrak}[1]{\ensuremath{{}\left.#1\right]}}
\providecommand{\brak}[1]{\ensuremath{\left(#1\right)}}
\providecommand{\lbrak}[1]{\ensuremath{\left(#1\right.}}
\providecommand{\rbrak}[1]{\ensuremath{\left.#1\right)}}
\providecommand{\cbrak}[1]{\ensuremath{\left\{#1\right\}}}
\providecommand{\lcbrak}[1]{\ensuremath{\left\{#1\right.}}
\providecommand{\rcbrak}[1]{\ensuremath{\left.#1\right\}}}
\providecommand{\ceil}[1]{\left \lceil #1 \right \rceil }
\theoremstyle{remark}
\newtheorem{rem}{Remark}
\newcommand{\sgn}{\mathop{\mathrm{sgn}}}
\providecommand{\abs}[1]{\left\vert#1\right\vert}
\providecommand{\res}[1]{\Res\displaylimits_{#1}} 
\providecommand{\norm}[1]{\left\lVert#1\right\rVert}
%\providecommand{\norm}[1]{\lVert#1\rVert}
\providecommand{\mtx}[1]{\mathbf{#1}}
\providecommand{\mean}[1]{E\left[ #1 \right]}
\providecommand{\fourier}{\overset{\mathcal{F}}{ \rightleftharpoons}}
%\providecommand{\hilbert}{\overset{\mathcal{H}}{ \rightleftharpoons}}
\providecommand{\system}{\overset{\mathcal{H}}{ \longleftrightarrow}}
	%\newcommand{\solution}[2]{\textbf{Solution:}{#1}}
\newcommand{\solution}{\noindent \textbf{Solution: }}
\newcommand{\cosec}{\,\text{cosec}\,}
\providecommand{\dec}[2]{\ensuremath{\overset{#1}{\underset{#2}{\gtrless}}}}
\newcommand{\myvec}[1]{\ensuremath{\begin{pmatrix}#1\end{pmatrix}}}
\newcommand{\mydet}[1]{\ensuremath{\begin{vmatrix}#1\end{vmatrix}}}
%\numberwithin{equation}{section}
\numberwithin{equation}{subsection}
%\numberwithin{problem}{section}
%\numberwithin{definition}{section}
\makeatletter
\@addtoreset{figure}{problem}
\makeatother

\let\StandardTheFigure\thefigure
\let\vec\mathbf
%\renewcommand{\thefigure}{\theproblem.\arabic{figure}}
%\renewcommand{\thefigure}{\theproblem}
\renewcommand{\thefigure}{\thesection}
%\setlist[enumerate,1]{before=\renewcommand\theequation{\theenumi.\arabic{equation}}
%\counterwithin{equation}{enumi}


%\renewcommand{\theequation}{\arabic{subsection}.\arabic{equation}}

\def\putbox#1#2#3{\makebox[0in][l]{\makebox[#1][l]{}\raisebox{\baselineskip}[0in][0in]{\raisebox{#2}[0in][0in]{#3}}}}
     \def\rightbox#1{\makebox[0in][r]{#1}}
     \def\centbox#1{\makebox[0in]{#1}}
     \def\topbox#1{\raisebox{-\baselineskip}[0in][0in]{#1}}
     \def\midbox#1{\raisebox{-0.5\baselineskip}[0in][0in]{#1}}

\vspace{3cm}

\title{
%	\logo{
वामन से परिचय
%	}
}
\author{ गाड़ेपल्लि वेंकट विश्वनाथ शर्मा $^{*}$% <-this % stops a space
	\thanks{*रचयिता भारतीय प्रौद्योगिकी संस्थान, हैदराबाद,५०२२८५ के विद्युत अभियान्त्रिकी विभाग में कार्यरत हैं, ईमेल:gadepall@ee.iith.ac.in। यह लेख मुक्त स्रोत विचारधारा के अनुरूप  है।}
	
}	
%\title{
%	\logo{Matrix Analysis through Octave}{\begin{center}\includegraphics[scale=.24]{tlc}\end{center}}{}{HAMDSP}
%}


% paper title
% can use linebreaks \\ within to get better formatting as desired
%\title{Matrix Analysis through Octave}
%
%
% author names and IEEE memberships
% note positions of commas and nonbreaking spaces ( ~ ) LaTeX will not break
% a structure at a ~ so this keeps an author's name from being broken across
% two lines.
% use \thanks{} to gain access to the first footnote area
% a separate \thanks must be used for each paragraph as LaTeX2e's \thanks
% was not built to handle multiple paragraphs
%

%\author{<-this % stops a space
%\thanks{}}
%}
% note the % following the last \IEEEmembership and also \thanks - 
% these prevent an unwanted space from occurring between the last author name
% and the end of the author line. i.e., if you had this:
% 
% \author{....lastname \thanks{...} \thanks{...} }
%                     ^------------^------------^----Do not want these spaces!
%
% a space would be appended to the last name and could cause every name on that
% line to be shifted left slightly. This is one of those "LaTeX things". For
% instance, "\textbf{A} \textbf{B}" will typeset as "A B" not "AB". To get
% "AB" then you have to do: "\textbf{A}\textbf{B}"
% \thanks is no different in this regard, so shield the last } of each \thanks
% that ends a line with a % and do not let a space in before the next \thanks.
% Spaces after \IEEEmembership other than the last one are OK (and needed) as
% you are supposed to have spaces between the names. For what it is worth,
% this is a minor point as most people would not even notice if the said evil
% space somehow managed to creep in.



% The paper headers
%\markboth{Journal of \LaTeX\ Class Files,~Vol.~6, No.~1, January~2007}%
%{Shell \MakeLowercase{\textit{et al.}}: Bare Demo of IEEEtran.cls for Journals}
% The only time the second header will appear is for the odd numbered pages
% after the title page when using the twoside option.
% 
% *** Note that you probably will NOT want to include the author's ***
% *** name in the headers of peer review papers.                   ***
% You can use \ifCLASSOPTIONpeerreview for conditional compilation here if
% you desire.




% If you want to put a publisher's ID mark on the page you can do it like
% this:
%\IEEEpubid{0000--0000/00\$00.00~\copyright~2007 IEEE}
% Remember, if you use this you must call \IEEEpubidadjcol in the second
% column for its text to clear the IEEEpubid mark.



% make the title area
\maketitle

\newpage

\tableofcontents


\bigskip

\renewcommand{\thefigure}{\theenumi}
\renewcommand{\thetable}{\theenumi}
\renewcommand{\abstractname}{सार}
\renewcommand{\nomname}{नामकरण}
\renewcommand{\solution}{हल: }
\renewcommand{\figurename}{आकृति.}
\renewcommand{\tablename}{सारणी.}
%\renewcommand{\theequation}{\theenumi}

%\begin{abstract}
%%\boldmath
%In this letter, an algorithm for evaluating the exact analytical bit error rate  (BER)  for the piecewise linear (PL) combiner for  multiple relays is presented. Previous results were available only for upto three relays. The algorithm is unique in the sense that  the actual mathematical expressions, that are prohibitively large, need not be explicitly obtained. The diversity gain due to multiple relays is shown through plots of the analytical BER, well supported by simulations. 
%
%\end{abstract}
% IEEEtran.cls defaults to using nonbold math in the Abstract.
% This preserves the distinction between vectors and scalars. However,
% if the journal you are submitting to favors bold math in the abstract,
% then you can use LaTeX's standard command \boldmath at the very start
% of the abstract to achieve this. Many IEEE journals frown on math
% in the abstract anyway.

% Note that keywords are not normally used for peerreview papers.
%\begin{IEEEkeywords}
%Cooperative diversity, decode and forward, piecewise linear
%\end{IEEEkeywords}



% For peer review papers, you can put extra information on the cover
% page as needed:
% \ifCLASSOPTIONpeerreview
% \begin{center} \bfseries EDICS Category: 3-BBND \end{center}
% \fi
%
% For peerreview papers, this IEEEtran command inserts a page break and
% creates the second title. It will be ignored for other modes.
%\IEEEpeerreviewmaketitle

\begin{abstract}
वामन एक सूक्ष्म नियंत्रण परिपथफलक है जो  प्राथमिक विद्यालयों से लेकर  विश्व विद्यालयों के सभी  छात्रों को एक सरल विधि से संगणक क्रमादेश,  यंत्रोपवस्तु एवं अयंत्रोपवस्तु
सीखने में सहायक है।  इस लेख के द्वारा  वामन से छात्रों का परिचय कराया जाएगा।


%ॐ श्री गणेशाय नमः॥
%\\
%\indent जय श्री राम।
%This manual provides a simple introduction to Digital Design.
\end{abstract}

%\section{Nomenclature}
\printnomenclature[1.7in]
%\addcontentsline{toc}{chapter}{Nomenclature}

%\nomenclature{Seven Segment Display}{ सप्तांश प्रदर्शी} 
%\nomenclature{Code}{ गूढ़} 
%\nomenclature{Decoder}{ गूढ़वाचक}
%\nomenclature{Incrementing}{ परवर्ती}
%\nomenclature{Decrementing}{ पूर्ववर्ती}
%\nomenclature{Karnaugh Map}{ कार्नो मानचित्र}
%\nomenclature{Implicant}{ विवक्षक}
%\nomenclature{LED}{ प्रकाश उत्सर्जक यंत्र}
%\nomenclature{Table}{ सारणी}
%\nomenclature{Figure}{ आकृति}
%\nomenclature{Variable}{ चर}
%\nomenclature{Input}{ आगत}
%\nomenclature{Output}{ निर्गत}
%\nomenclature{Boolean Algebra}{ बूलीय बीजगणित}
%\nomenclature{Verify}{ सत्यापित}
%\nomenclature{Combinational Logic}{ संयोजक तर्क}
%\nomenclature{Minimize}{ कनिष्ठीकरण}
\nomenclature{Execute}{ निष्पादित, चालयन}
%\nomenclature{Expression}{ व्यंजक}
%\nomenclature{Equation}{ समीकरण}
%\nomenclature{Axiom}{ अभिगृह}
%\nomenclature{Reduced}{ समानयनिक}
%\nomenclature{C Program}{ C क्रमादेश}
\nomenclature{Programming}{  क्रमादेशन}
%\nomenclature{Frequency}{ आवृत्ति}
%\nomenclature{Bit length}{ मात्राभार}
%\nomenclature{Cable}{ रज्जु}
%\nomenclature{Button}{ गण्ड}
%\nomenclature{Left}{ वाम}
%\nomenclature{Right}{दक्षिण}
%\nomenclature{Blink}{श्मील}
%\nomenclature{IP Address}{अनिकेत}
%\nomenclature{Send}{प्रेषण}
%\nomenclature{File}{सञ्चिका}
%\nomenclature{Setup}{सप्रतिष्ठान}
%\nomenclature{Execute}{चालयन}
%\nomenclature{Interval}{अंतराल}

%\nomenclature{Binary}{ द्विआधारी}
%\nomenclature{Truth Table}{सत्य सारणी}
%\nomenclature{Derive}{व्युत्पन्न }
%\nomenclature{XOR}{अर्ध योग}
%\nomenclature{Complement}{पूरक}
%\nomenclature{Decade Counter}{दशक गणित्र}
%\nomenclature{Digital Logic}{अंकीय तर्क}
%\nomenclature{Function}{फलन}
%\nomenclature{Block Diagram}{खण्डारेख}
%\nomenclature{Circuit}{परिपथ}
%\nomenclature{Simplified Expression}{सरलीकृत व्यंजक}
%\nomenclature{Delay}{अतिकाल}
%\nomenclature{Minute}{निमिश}
%\nomenclature{Synchronous}{तुल्यकालिक}
%\nomenclature{Finite State Machine}{परिमित अवस्था यंत्र}
%\nomenclature{State Transition Table}{अवस्थान्तरण सारणी}
%\nomenclature{Flip flop}{द्विविध}
%\nomenclature{Design}{अभिकल्प}
%\nomenclature{Implementation}{कार्यान्वयन}
%\nomenclature{Period}{आवर्त}
\nomenclature{Hardware}{यंत्रान्श}
\nomenclature{Software}{तंंत्रान्श}
\nomenclature{Computer}{संगणक}
%\nomenclature{Board}{परिपथफलक}
%\nomenclature{Port}{पत्तन}
\nomenclature{Now}{इदान}
\nomenclature{Weblink}{जालबन्धन}
\nomenclature{Download}{अवाहरत}






 


%
\section{सप्रतिष्ठान}
\renewcommand{\theequation}{\theenumi}
\renewcommand{\thefigure}{\theenumi}
\begin{enumerate}[label=\thesection.\arabic*.,ref=\thesection.\theenumi]
\numberwithin{equation}{enumi}
\numberwithin{figure}{enumi}
\numberwithin{table}{enumi}

\item वामन एवं रास्पबेरी पै  को यूएसबी रज्जु से योजित करें।
\item योजित स्थल के निकट वाम दिशा में एक प्रकाश उत्सर्जक यंत्र एवं गण्ड  उपस्थित है।  वहीं दक्षिण में एक और गण्ड है।  
\item दक्षिण गण्ड को दबाकर शीघ्र वाम गण्ड को दबायें।  हरित दीप श्मीलित होते हुए दिखेगा।  यह संकेत हैं कि वामन
क्रमादेश के लिए सिद्ध है।

\item अब एंड्राइड यंत्र में टेरमक्स-उबुन्टु में प्रवेश करें ऐवं निम्न आदेशों का चालयन करें।
\begin{lstlisting}
cd /storage/emulated/0/Download
svn co https://github.com/gadepall/pygmy/trunk/installation/blink
ql_symbiflow -compile -src /storage/emulated/0/Download/blink -d ql-eos-s3 -P PU64 -v helloworldfpga.v -t helloworldfpga -p quickfeather.pcf -dump binary
scp /storage/emulated/0/Download/blink/helloworldfpga.bin pi@192.168.0.114:
\end{lstlisting}
ऊपर helloworldfpga.bin सञ्चिका प्रेषण पूर्व रास्पबेरी पै का उचित अनिकेत दें। 
\item अब रास्पबेरी पै  में प्रवेश कर निम्न  आदेश का चालयन करें
\begin{lstlisting}
python3 /root/pygmy-dev/pygmy-sdk/TinyFPGA-Programmer-Application/tinyfpga-programmer-gui.py --port /dev/ttyACM0 --appfpga /home/pi/helloworldfpga.bin --mode fpga
\end{lstlisting}
\item उपरोक्त आदेश में उचित पत्तन दें। तत्पश्चात यूएसबी पत्तन के दक्षिण दिशा में गण्ड को दबायें। कुछ समय पश्चचात प्रकाश उत्सर्जक रक्तिम रंग में श्मीलित होगा। 
\end{enumerate}
%
\section{आवृत्ति}

\renewcommand{\theequation}{\theenumi}
\renewcommand{\thefigure}{\theenumi}
\begin{enumerate}[label=\thesection.\arabic*.,ref=\thesection.\theenumi]
\numberwithin{equation}{enumi}
\numberwithin{figure}{enumi}
\numberwithin{table}{enumi}
\item निम्न वेरिलॉग क्रमादेश
\label{सम:क्रमादेश}
\begin{lstlisting}
codes/blink/helloworldfpga.v
\end{lstlisting}
की  इन पङ्क्तियों पर ध्यान दें । 
%
\begin{lstlisting}
delay = delay+1;                                                                                                   
if(delay > 20000000)
begin
delay=27'b0;
led=!led;
end
\end{lstlisting}
%
इससे हम ज्ञात कर सकते हैं की वामन के दीप की श्मीलनावृत्ति  20 MHz है ।  
\item निर्देश \ref{सम:क्रमादेश} में 
\label{सम:द्विआधार}
\begin{lstlisting}
if(delay > 20000000)
\end{lstlisting}
%
को 
\begin{lstlisting}
if(delay==27'b1001100010010110100000000)
\end{lstlisting}
से प्रतिस्थापित कर क्रमादेश का चालयन करें ।  
\item यह विदित है कि यदि आवृत्ति 20 MHz है तो अतिकाल 1 निमिश है ।   
वेरिलॉग क्रमादेश में संशोधन कर अतिकाल को 0.5 निमिश करें । 
\item 20 MHz का  मात्राभार  ज्ञात कीजिये।
\\
\solution 
\begin{align}
\log_2\brak{20000000} \approx 27
\end{align}
\item पायथन क्रमादेश लिख कर उक्त उत्तर प्राप्त करें।
\\
\solution निम्न क्रमादेश का चालयन कर निर्देश \ref{सम:द्विआधार} से तुलना करें । 
\begin{lstlisting}
codes/blink/freq_count.py
\end{lstlisting}
\item निर्देश \ref{सम:क्रमादेश} में प्रदत्त क्रमादेश में निम्न पङ्क्ति 
\begin{lstlisting}
assign redled = led; //If you want to change led colour to red,
\end{lstlisting}
को 
\begin{lstlisting}
assign blueled = led; 
\end{lstlisting}
से प्रतिस्थापित कर निष्पादित करें । 
\end{enumerate}
\end{document}



